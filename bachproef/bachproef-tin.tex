\documentclass{bachproef-tin}

\usepackage{hogent-thesis-titlepage} % Titelpagina conform aan HOGENT huisstijl
\usepackage{graphicx,caption,subfig,tabularx}
\usepackage{floatrow,enumitem,csquotes,pgfplots}

%%---------- Documenteigenschappen ---------------------------------------------


% De titel van het rapport/bachelorproef
\title{Een bedrijfsnetwerk beter beveiligen en beheren met behulp van Cisco Identity Services Engine}


\author{Joeri Verhavert}


\promotor{Olivier Rosseel}


\copromotor{Dean De Blieck}

\instelling{Axians - Fit IT}

% Academiejaar
\academiejaar{2019-2020}

% Examenperiode
%  - 1e semester = 1e examenperiode => 1
%  - 2e semester = 2e examenperiode => 2
%  - tweede zit  = 3e examenperiode => 3
\examenperiode{2}

%===============================================================================
% Inhoud document
%===============================================================================

\begin{document}



%---------- Titelblad ----------------------------------------------------------
\inserttitlepage

%---------- Samenvatting, voorwoord --------------------------------------------
\usechapterimagefalse
%%=============================================================================
%% Voorwoord
%%=============================================================================

\chapter*{\IfLanguageName{dutch}{Woord vooraf}{Preface}}
\label{ch:voorwoord}

Met het schrijven van dit voorwoord leg ik het laatste hand aan mijn eindwerk. Het was een periode waarin ik heel veel heb bijgeleerd, vooral op business, maar ook op persoonlijk vlak.  Graag maak ik dan ook van de gelegenheid gebruik om een aantal mensen te bedanken want zonder hen was dit nooit gelukt. 
\newline
\newline
In de eerste plaats wil ik mijn promotor, de heer Olivier Rosseel, bedanken. Dankzij de heer Rosseel kwam dit eindwerk in de goede banen terecht. Steeds toonde u mij de richting waar ik heen moest. Daarnaast nam u telkens opnieuw de tijd om mijn draft versies te overlopen en te controleren. Hiervoor een welgemeende dank!
\newline
\newline
Vervolgens wil ik graag mijn collega’s van het stagebedrijf Axians bedanken voor de fijne samenwerking en de hulp die ik heb gekregen. Jullie hebben mij enorm gesteund en waren steeds bereid om mij te helpen. Bedankt!
\newline
\newline
Mevrouw Cousy, u nam in uw drukke agenda de tijd om dit eindwerk zorgvuldig na te lezen op spelling en op taal. Zonder u was deze bachelorproef een taalfiasco. Bedankt!
\newline
\newline
Tot slot wil ik graag mijn ouders bedanken voor de vele inspanningen die zijn hebben geleverd in de afgelopen jaren. Jullie stonden steeds paraat na alle ups maar ook na alle downs. Dankuwel mama en papa!
\newline
\newline
Voor u ligt mijn eindwerk, een resultaat van wekenlang hard werken. Ik hoop dat het resultaat van mijn werk zichtbaar mag zijn. 
\newline
\newline
Ik wens u veel leesplezier.
\newline
Joeri Verhavert







\chapter*{\IfLanguageName{dutch}{Samenvatting}{Abstract}}
\label{ch:Samenvatting}

Dit onderzoek kan dienen om bedrijfsnetwerken beter te beheren en te beveiligen tegen interne of externe gevaren met behulp van Cisco Identity Services Engine. Dit omdat bedrijfsnetwerken steeds meer nood hebben aan een network access control product die netwerken beter kunnen beheren en beveiligen. Dit onderzoek focust zich voornamelijk op implementatie en evaluatie van Cisco Identity Services Engine met use cases '\textit{Port-based, Policy-based en Tread-Centric network access control}'. 
\newline
\newline
Daarnaast worden de resultaten van de enquête tijdens dit onderzoek geanalyseerd die op het einde mee verwerkt zijn met de evaluatie van Cisco Identity Services Engine en zijn use cases. In de literatuurstudie zijn twee andere network access control producten mee verwerkt om aan te tonen dat integratie van deze network access control producten ook mogelijk zijn om een netwerk robuster te maken. Bij de analyses wordt een netwerk geanalyseerd voor integratie van de use cases en eens na de integratie van de use cases. Uit deze twee vergelijkingen is een evaluatie opgesteld. In dit geschrift vindt u een inleiding tot het onderwerp dat verwerkt is in de literatuurstudie. Daarnaast vindt u de resultaten van de uitgevoerde testen en van de enquête in Hoofdstuk \ref{ch:Resultaten}. 
\newline
\newline
Uit dit onderzoek blijkt dat integratie van Cisco Identity Services Engine de vruchten plukt op beheersbaarheid en beveiliging, dit wordt ook bevestigd in de resultaten van de enquête. Vervolgens komt uit de enquête naar voor dat \textit{'Identity-based network access control'} voor vakspecialisten de belangrijkste use case is binnen het Cisco Identity Services Engine product. Toekomstig onderzoek kan over \textit{'Identity-based network access control'} uitgevoerd worden wat de voordelen van deze use case naar boven brengt.


%---------- Inhoudstafel -------------------------------------------------------
\pagestyle{empty} % Geen hoofding
\tableofcontents  % Voeg de inhoudstafel toe
\cleardoublepage  % Zorg dat volgende hoofstuk op een oneven pagina begint
\pagestyle{fancy} % Zet hoofding opnieuw aan

%---------- Lijst figuren, afkortingen, ... ------------------------------------


\listoffigures
\listoftables
%%=============================================================================
%% Inleiding
%%=============================================================================

\chapter{\IfLanguageName{dutch}{Inleiding}{Introduction}}
\label{ch:inleiding}

\begin{displayquote}
	“It takes 20 years to build a reputation and few minutes of cyber-incident to ruin it.” – Stephane Nappo
\end{displayquote}

De quote van Stephane Nappo, Global Chief Information Security Officer, expliceert met zijn quote het belang van cybersecurity in een notendop. Cybersecurity is namelijk één van de meest besproken technologische onderwerpen die centraal staan in bedrijven die 'oorlog' voeren met cyber criminelen. Information and Communications Technology wordt de dag vandaag alsmaar grootschaliger toegepast en zal in de nabije toekomst zeker niet verminderen. Door de grote uitbreiding in Information and Communications Technology stijgt de kans op cyber aanvallen evenredig mee. Persoonlijke gegevens of resources kunnen door de stijging in cyber aanvallen sneller gestolen worden dan 20 jaar geleden. Een oplossing was dus hoognodig, waardoor het cybersecurity begrip ontstond. 
\newline
\newline
Daarbij moeten organisaties sterk investeren om de aanvalspogingen drastisch te doen verminderen om zo de cyber criminelen een stapje voor te zijn. Nieuwe principes zoals 'Internet of Things' en 'Bring Your Own Device' maken bedrijven het er niet gemakkelijker op. Tegenwoordig worden 'Internet of Things' en 'Bring Your Own Device' principes bijna in alle bedrijven toegepast. Bedrijven moeten met als gevolg de proliferatie netwerk-compatibele apparaten beter ondersteunen, zelfs wanneer een groot aantal cyber bedreigingen en veel gepubliceerde datalekken aantonen dat het belang van netwerk bescherming van cruciaal belang is. 
\newline
\newline
Eind apparaten worden door werknemers frequent mee naar huis genomen wanneer principes zoals 'Internet of Things' en 'Bring Your Own Device' zijn toegepast in organisaties. Hierdoor maken de reizende eind apparaten het netwerk veel kwetsbaarder, waardoor men vaak ook niet weet wie gebruik maakt van het netwerk. Vanuit dit probleem kwam Axians met de vraag voor dit onderzoek, waarbij het gebruik van Cisco Identity Services Engine in samen werking met 'Policy-based access control', 'Port-based access control' en Thread-Centric access control centraal staat. Cisco Identity Services Engine is een network access control product dat de handhaving van beveiligings- en toegangsbeleid mogelijk maakt voor eind apparaten die zijn aangesloten op het netwerk van het bedrijf. Het doel is om identiteitsbeheer en om de veiligsbeheer op verschillende eind apparaten en applicaties te vereenvoudigen. 
\newline
\newline
Hieruit is de onderzoeksvraag '\textit{Een bedrijfsnetwerk beter beveiligen en beheren met behulp van Cisco Identity Services Engine}' ontstaan, waarbij volgende doelstellingen mee bedacht zijn: 

\begin{itemize}
	\item Implementatie van Cisco Identity Services Engine in een netwerk.
	\item Integratie van 'Port-based access control met Cisco Identity Services Engine in een netwerk.
	\item Integratie van 'Policy-based access control met Cisco Identity Services Engine in een netwerk.
	\item Integratie van 'Thread-Centric access control met Cisco Identity Services Engine in een netwerk.
	\item Opstellen van een enquête over Cisco Identity Services Engine in een netwerk.
	\item Evaluatie van Cisco Identity Services Engine in een netwerk.
\end{itemize}

De samenleving en de Information and Communications technologieën zullen steeds verder evolueren waardoor ondernemingen in staat moeten om cybersecurity steeds te kunnen opvolgen. Men zal constant moeten bijleren om de strijd tegen cyberaanvallen te blijven winnen. De opvolging wordt er zeker en vast niet gemakkelijker op, maar het zal veel efficiënter verlopen als men dit in team kan doen. Cybersecurity vereist dus een samenwerking om de resources en gegevens te beschermen.

\section{\IfLanguageName{dutch}{Probleemstelling}{Problem Statement}}
\label{sec:probleemstelling}
Aangezien Information and Communications Technology centraal staat in de 21ste eeuw kan iedereen in aanraking komen met cyber aanvallen. Denk maar aan de talrijke pogingen die cyber criminelen via Facebook, Twitter, E-mail, enz. proberen om accounts, geld, informatie te bemachtigen bij verschillende slachtoffers. Veel van deze gevallen gebeuren op individueel vlak, maar bedrijven hebben evenveel kans om bestolen te worden. Men kan verschillende methodes bedenken hoe organisaties het slachtoffer kunnen zijn van cyberaanvallen. Maar dit onderzoek focust zich op principes zoals 'Internet of Things' en 'Bring Your Own Device'.
\newline
\newline
Hierdoor wordt het doelgroep van dit onderzoek beperkt met oog op de eind apparaten die het netwerk steeds verlaten en opnieuw binnen komen, specifiek zal het onderzoek zich richten op de eind apparaten van de werknemers van Axians die gebruiken maken van het netwerk. Aangezien deze eind apparaten het netwerk flink kunnen toetakelen.


\section{\IfLanguageName{dutch}{Onderzoeksvraag}{Research question}}
\label{sec:onderzoeksvraag}
Een veiliger en beter beheerbaar netwerk is wat men wil bereiken door het gebruik van een network access control product en use cases zoals :'\textit{Port-based access control, Policy-based access control}' en '\textit{Thread-Centric access control}'. Het network access control product dat uitwerkt wordt noemt men Cisco Identity Services Engine.

De hoofdonderzoeksvraag in deze bachelorproef is: 
\begin{displayquote}
	Hoe kunnen we een evoluerend bedrijfsnetwerk beter gaan beveiligen door het gebruik van Cisco Identity Services Engine?
\newline
\newline
\end{displayquote}
Verder zal deze bachelorproef zich verdiepen in het gebruik van de Cisco Identity Services Engine use cases. Hieronder zijn de deelonderzoeksvragen vernoemd die ook tot het onderzoek behoren.
\begin{displayquote}
	Welke invloed heeft de integratie van \textit{Port-based access control} in een netwerk?  
\newline
\newline
	Wat is meerwaarde van de integratie van \textit{Policy-based access control} in een netwerk?  
\newline
\newline
	Kan het netwerk wel degelijk beschermd worden tegen verdere uitbreiding van Malware door integratie van \textit{Thread-Centric access control} in een netwerk?  
\end{displayquote}

\section{\IfLanguageName{dutch}{Onderzoeksdoelstelling}{Research objective}}
\label{sec:onderzoeksdoelstelling}

%Wat is het beoogde resultaat van je bachelorproef? Wat zijn de criteria voor succes? Beschrijf die zo concreet mogelijk. Gaat het bv. om een proof-of-concept, een prototype, een verslag met aanbevelingen, een vergelijkende studie, enz.
Door middel van het Cisco Identity Services Engine network access control product zal dit onderzoek inzicht bieden op de beveiligen en het beter beheer van netwerken tegen interne of externe gevaren. Vervolgens moeten de resultaten van de enquête een eenduidig antwoord bieden die in lijn liggen met de resultaten van de uitgevoerde testen, enkel dan is dit deelonderzoek geslaagd. Door de uitvoering van dit onderzoek wil men dat ondernemingen stappen zetten in de juiste richting tegen de strijd met cyber criminelen. Zo kan informatie, geld en andere resources binnen bedrijven steeds gewaarborgd blijven. Wanneer de bedrijven de waarde van een network access control product zoals Cisco Identity Services Engine inzien, dan is dit onderzoek volledig geslaagd.

\section{\IfLanguageName{dutch}{Opzet van deze bachelorproef}{Structure of this bachelor thesis}}
\label{sec:opzet-bachelorproef}

% Het is gebruikelijk aan het einde van de inleiding een overzicht te
% geven van de opbouw van de rest van de tekst. Deze sectie bevat al een aanzet
% die je kan aanvullen/aanpassen in functie van je eigen tekst.

De rest van deze bachelorproef is als volgt opgebouwd:

In Hoofdstuk~\ref{ch:Literatuurstudie} wordt een overzicht gegeven van de stand van zaken binnen het onderzoeksdomein, op basis van een literatuurstudie.

In Hoofdstuk~\ref{ch:methodologie} wordt de methodologie toegelicht en worden de gebruikte onderzoekstechnieken besproken om een antwoord te kunnen formuleren op de onderzoeksvragen.

In Hoofdstuk~\ref{ch:Proof of concept} wordt de proof of concept van de omgeving toegelicht en worden de gebruikte tools voor implemenatie van Cisco Identity Services Engine besproken. 

In Hoofdstuk~\ref{ch:Resultaten} worden de resultaten van de uitgevoerde testen en van de enquête toegelicht.

In Hoofdstuk~\ref{ch:conclusie}, tenslotte, wordt de conclusie gegeven en een antwoord geformuleerd op de onderzoeksvragen. Daarbij wordt ook een aanzet gegeven voor toekomstig onderzoek binnen dit domein.
\input{literatuurstudie}
\input{methodologie}
%%=============================================================================
%% Methodologie
%%=============================================================================

\chapter{\IfLanguageName{dutch}{Proof of concept}{Proefafdruk}}
\label{ch:Proof of concept}

Het doel van dit hoofdstuk is om een idee te creëren van de omgeving waarin Cisco Identity Services Engine is geïmplementeerd. Hierbij is een uitgebreide uitleg over de elementen die aanwezig zijn binnen de omgeving van Cisco terug te vinden in dit hoofdstuk. Ieder hardware apparaat is zorgvuldig beschreven, onder andere is er informatie te vinden over de specificatie, de configuraties en de instellingen van de hardware componenten. Daarnaast zijn ook de implementaties van de virtuele machines verder toegelicht.
\newline
\newline
Samen met deze informatie zijn belangrijke schema’s verwerkt in dit hoofdstuk, die een visueel beeld geven over dit afgezonderd netwerk. 
Vervolgens is er de sectie die de enquête bespreekt. De bespreking van de enquête bevat voornamelijk informatie die terug te vinden is in de sectie \ref{sec:enquête}. 

\section{Cisco Identity Services Engine omgeving}

Voor de implementaties en testen van Cisco Identity Services Engine en zijn benodigde use cases is er een omgeving vereist. Deze omgeving is voorzien door een Belgische aanbieder van digitale televisie, breedband-internet, mobiele telefonie, gekend als Telenet Group N.V. Implementaties en de testen van Cisco Identity Services Engine en met zijn benodigde use cases is uitgewerkt in een Telenet thuisnetwerk.
\newline
Zoals elk thuisnetwerk dat voorzien is door een Belgische internet aanbieder bevat het thuisnetwerk een aantal Telenet Access Points en een Telenet modem. De kabelverbinding die data verzend vanuit de Telenet centrale naar de Telenet modem is een coaxkabel. Het is dankzij de coaxkabel dat het thuisnetwerk verbonden is met het Wide Area Network, ook gekend als WAN.
\newline
\newline
Om de essentiële componenten van dit thuisnetwerk te beschermen tegen breuk, is de implementatie verder uitgewerkt in een afgezonderd netwerk. Dit wordt mogelijk gemaakt met behulp van een virtuele opensource router/firewall zoals Pfsense. Dit virtuele routertje dient als gateway voor de verschillende vlan’s waarbij het thuisnetwerk gebruikt zal worden als ‘uplink’ naar het internet. 



\subsection{Server VMware ESXi}
Het afgezonderd netwerkje bestaat uit een rack server waarop een viertal virtuele machines draaien. Deze virtuele machines zijn: Cisco Identity Service Engine, het virtuele routerje (Pfsense) en twee  Windows Server 2019 datacenters machines. Dit wordt eenvoudig weergegeven in figuur \ref{fig:vms} die de virutele machines weergeeft. De creatie van deze virtuele machines is mogelijk gemaakt door VMware ESXi. VWware ESXi is een type-1 hypervisor van enterprise-klasse, ontwikkeld door VMware voor het inzetten en bedienen van virtuele machines. 
\newline
\newline
De ESXi server is geconfigureerd door het gebruik van een ‘bootable usb’. Deze bootable usb bevatte al nodige software om de server te voorzien met VMware ESXi. Vervolgens werd de server geconfigureerd met een aantal belangrijke settings, zoals het instellen van een IP-adres, subnet mask, raid controller, enz. Waarbij zijn volgende settings voorzien op deze ESXi server: 

\begin{itemize}
	\item IP adres: 192.168.0.183
	\item Subnet: 255.255.255.0
	\item Default gateway: 192.168.0.1
	\item raid controller: RAID 10
\end{itemize}

\begin{figure}[H]
	\centering
	\includegraphics[height=0.3\textheight]{servervm.png}
	\caption{Virtuele machines in VMware ESXi server}
	\label{fig:vms}
\end{figure}

\newpage
Door de configuratie van deze IP instellingen, is de VMware ESXi server toeganglijk via de webbrowser. Virtuele machines kunnen via de webbrowser beheerd, verwijderd en gecreëerd worden. Daarnaast werd de hard disk drive ingesteld met een RAID controller. RAID, een afkoring voor Redundant array of independent disks is een dataopslagtechnologie waarbij meerdere harde schijven gecombineerd worden tot één of meer logische virtuele opslageenheden. Dit heeft als doel om de veiligheid, snelheid en de capaciteit te kunnen vergroten. Hierbij is op de VMware ESXi RAID 10 geconfigueerd. RAID 10 is een hybride combinatie tussen RAID 1 en RAID 0. Waarbij men de snelheid van striping met de veiligheid van mirroring combineert. 
\newline
\newline
Dit is de veiligste en snelste methode maar ook het duurste. Het is een dure methode omdat er gebruik wordt gemaakt van RAID 1. Er is dus voor iedere 1TB aan opslagruimte ook 1TB aan mirror ruimte nodig, in combinatie met RAID 0 waardoor er veel fysieke harde schijven nodig zijn. In figuur \ref{fig:Raid10} wordt een voorbeeld van een RAID 10 schema weergegeven.

\begin{figure}[H]
	\centering
	\includegraphics[height=0.25\textheight]{Raid10.png}
	\caption{Voorbeeld configuratieschema RAID 10 (\cite{Raid10}).}
	\label{fig:Raid10}
\end{figure}

Vervolgens zijn er in de VMware ESXi server opstelling drie van de vier fysieke ethernet adapters gebruikt. Deze adapters zijn op hun beurt verbonden met een virtuele switch. Elke fysieke adapters heeft een andere functie die weergegeven zijn in de onderstaande lijst. 


\begin{itemize}
	\item vmnic0,is de fysieke adapter die verbonden is met de virtuele switch, genaamd 'VSwitch0'.
	\item vmnic2,is de fysieke adapter die verbonden is met de virtuele switch, genaamd 'Sub\textunderscore switch'.
	\item vmnic3,is de fysieke adapter die verbonden is met de virtuele switch, genaamd 'Cisco \textunderscore switch'.
\end{itemize}

\newpage
\subsubsection{Virtuele switches}
\subsubitem{\bf VSwitch0}
\newline
De 'VSwitch0' is een virtuele switch die voorzien is om de VMware ESXi server te contacteren via de webbrowser. Op figuur \ref{fig:Vswitch0} ziet men dat de 'VSwitch0' is ingesteld met vlan id 0, dat geen vlan identificatie voorziet. 
\newline
\newline
TCP/IP pakketen passeren deze interface wanneer gebruikers surfen naar "https://192.168.0.183". Vervolgens passeert de TCP/IP pakketen langs vmnic0 die de informatie doorgeeft aan de VSwitch0. Wanneer de informatie terecht komt op de VSwitch0, stuurt hij op zijn beurt dit door naar de VMKernel poort.


\begin{figure}[H]
	\centering
	\includegraphics[width=0.7\textwidth]{VSwitch0.png}
	\caption{Topologie van de 'VSwitch0'}
	\label{fig:Vswitch0}
\end{figure}

\subsubitem{\bf Sub\textunderscore switch}
\newline
Op figuur \ref{fig:subswitch} ziet men dat de 'Sub\textunderscore switch' een virtuele switch is die gebruikt wordt voor de WAN interface van de Pfsense. De Wide Area Network interface voorziet data overdracht van het thuisnetwerk naar het afgezonderd LAN netwerk. Hierdoor is connectie met het afgezonderd netwerk mogelijk via deze interface. Bovendien wordt deze virtuele switch ook gebruikt door de jumphost om remote desktop protocol connecties te maken met de virtuele machines binnen het afgezonderd LAN netwerk. 

\begin{figure}[H]
	\centering
	\includegraphics[width=0.7\textwidth]{Subswitch.png}
	\caption{Topologie van de 'Sub\textunderscore switch'}
	\label{fig:subswitch}
\end{figure}
\newpage
\subsubitem{\bf Cisco\textunderscore switch}
\newline
Tot slot is er de 'Cisco\textunderscore switch' die gebruikt wordt om connectie te maken met alle apparaten, achterliggend de fysieke Cisco switch. Wanneer eind apparaten met de Cisco switch verbinden, zal de data passeren via de 'Cisco\textunderscore switch'. Op de figuur \ref{fig:Ciscoswitch} is ook te zien dat alle virtuele machines zich in deze omgeving bevinden. Dit zorgt ervoor dat de Cisco\textunderscore switch ingesteld is met vlan id 10, waarbij enkel data vanuit vlan 10 wordt doorgestuurd naar de voorziene eind apparaten.

\begin{figure}[H]
	\centering
	\includegraphics[width=0.7\textwidth]{CiscoSwitch_vmware.png}
	\caption{Topologie van de 'Cisco\textunderscore switch'}
	\label{fig:Ciscoswitch}
\end{figure}

\subsubsection{Server ESXi specificaties}
De VMware ESXi server is van het merk International Business Machines Corporation, gekend als IBM. Deze server werd oorsprongelijk gebruikt in de productie omgeving van Axians, maar wordt sinds kort als test server gebruikt. In figuur \ref{fig:VmwareSer} ziet men een afbeelding van de IBM ESXi server.
\newline
\newline
Het model van de VMware ESXi server is ‘System x3350 M3’ dat gekend staat binnen Axians als ‘de oude shr-esx-04 server’. De IBM ‘System x3350 M3’ bevat volgende specificaties:


\begin{itemize}
	\item Vormfactor: 1U Rack
	\item Processor:
	\begin{itemize}
		\item Proccessorsnelheid: 2.53GHz
		\item Processor: 6-core processor
	\end{itemize}
	\item Geheugen:
	\begin{itemize}
		\item Intern geheugen: 240 GB Random Access Memory
		\item Intern geheugentype: Double Data Rate 3 Synchronous Dynamic Random-Access Memory, gekend als DDR3 SDRAM
	\end{itemize}
	\item Opslag:
	\begin{itemize}
		\item Maximum aantal schrijven: 8 Slots
		\item Opslag: 4 Terabyte HHD 7500 RPM
	\end{itemize}
	\item Netwerk
	\begin{itemize}
		\item Ethernet interface type: Gigabit Ethernet
		\item Aantal ethernet poorten: 4 
	\end{itemize}
\end{itemize}

\begin{figure}[H]
	\centering
	\includegraphics[width=0.5\textwidth]{serveresxi.png}
	\caption{Foto van de ESXi VMware server}
	\label{fig:VmwareSer}
\end{figure}

\subsection{Cisco switch}
Naast de VMware ESXi server, bestaat de Cisco Identity Services Engine omgeving ook uit een Cisco switch. Op figuur \ref{fig:switch} ziet men een foto van de Cisco switch. Dankzij een aantal configuraties op de Cisco switch kan de Cisco Identity Services Engine virtuele machine communiceren met de fysieke switch. Deze noodzakelijke configuraties speelden een belangrijke rol in de configuratie van de 'Port-based network access control' use case. 
\newline
\newline
Door een aantal basis configuraties op deze switch uit te voeren, wordt communicatie tussen het afgezonderd netwerk en de eind apparaten die zich achter de Cisco switch bevinden ook mogelijk. Deze informatie is terug te vinden in de sectie \ref{sec:config}.


\subsubsection{Configuratie}
\label{sec:config}
Als eerst zijn een aantal commando's uitgevoerd die het netwerk apparaat beveiligen met een secret, line passwords en password encryption om inbreuk tegen te gaan. Vervolgens werd Gigabit Ethernet 0/1 geconfigureerd met de gepaste trunk settings om de gobale communicatie mogelijk te maken. Een trunk configuratie maakt data overdracht van de verschillende vlan's mogelijk. Volgende twee trunk commando's zijn hiervoor uitvoerd:  

\begin{itemize}
	\item switchport mode trunk
	\item switchport trunk allow vlan 10 
\end{itemize}

Gigabit Ethernet 0/1 werd rechtsverbonden met één van de interfaces op de VMware ESXi server, waarbij de virtuele switch ingesteld werd met vlan id 0 dat gelijk staan aan geen vlan identificatie. Vervolgens werd
op de Cisco switch vlan 10 gecreeërd met ip adres '192.168.17.5' en met subnet mask '255.255.255.0'. Vermits de interface Gigabit Ethernet 0/1 geconfigureerd is om verbinding te maken met de VMware ESXi server, zijn de overige interfaces bedoeld voor de communicatie met de eind apparaten. 
\newline
\newline
Om de 'Port-based network access control' use case te implementeren, werd een nieuw AAA model ingesteld. Dit AAA model werd vervolgens geïnitialiseerd met een radius server die gelijk staat aan het Internet Protocol adres van Cisco Identity Services Engine. Ten slotte werd de dot1x aaa authentication methode met zijn standaard netwerk groep mee configureerd. 
\newline
\newline
Om de configuraties van de Cisco switch te beëindeigen, werd dot1x system-auth-control vastgelegd en werden alle poorten voorzien van de nodige 'Port-based access control' configuraties. Figuur \ref{fig:RunningConfig} toont de running config van de Switch Cisco, daarnaast zijn alle nodige commando's in onderstaande lijst ook weergegeven.

\begin{itemize}
	\item \#enable
	\item \#config t
	\item (config)\#aaa new-model
	\item (config)\#aaa group server radius ISE
	\item (config-sg-radius)\#server-private 192.168.17.4 key Admin2020
	\item (config-sg-radius)\#exit
	\item (config)\#aaa authentication dot1x
	\item (config)\#aaa authorization network default group ISE
	\item (config)\#dot1x system-auth-control
	\item (config)\#interface range gig0/2-12
	\item (config-if-range)\#switchport mode access
	\item (config-if-range)\#switchmode access vlan 10
	\item (config-if-range)\#authentication host-mode multi-host
	\item (config-if-range)\#authentication port-control auto
	\item (config-if-range)\#dot1x pae auth
	\item (config-if-range)\#end
\end{itemize}

\begin{figure}[H]
	\centering
	\subfloat{{\includegraphics[width=3cm]{Switch_config1.png} }}%
	\qquad
	\subfloat{{\includegraphics[width=3cm]{Switch_config2.png} }}%
	\qquad
	\subfloat{{\includegraphics[width=3cm]{Switch_config3.png} }}%
	\caption{Running config van de Cisco switch}%
	\label{fig:RunningConfig}%
\end{figure}
Verder zijn 1G cat5e ethernet kabels gebruikt om de hardware componenten met elkaar te laten communiceren. De Unshielded Twisted Pair Ethernet kabels halen snelheden tot 1000 Mbit/s met een doorvoersnelheid van 100mhz. Wat geschikt is voor dit afgezonderd netwerk.

\subsubsection{Cisco switch specificaties}
De Cisco switch behoord tot de ‘Catalyst 2960-CX’ series die nu gekend staat binnen Axians als de ‘Demo switch’. Hierbij vindt men in onderstaande lijst de specificaties van de Cisco Catalyst 2960-CX serie switch:

\begin{itemize}
	\item Power over Ethernet:
	\begin{itemize}
		\item Ondersteunend: Ja
		\item Standaard: 802.3at (PoE+)
	\end{itemize}
	\item Netwerk:
	\begin{itemize}
		\item 8 Gigabit Ethernet aansluitingen
		\item 2 Gigabit Ethernet Copper uplinks
		\item 2 Gigabit Ethernet Small form-factor pluggable uplinks, gekend als SFP
	\end{itemize}
	\item Besturingsysteem/software:
	\begin{itemize}
		\item OS: Cisco IOS
    \end{itemize}
\end{itemize}

\begin{figure}[H]
	\centering
	\includegraphics[width=0.5\textwidth]{ciscoswitch.png}
	\caption{Foto van de Cisco switch}
	\label{fig:switch}
\end{figure}

\subsection{Virtuele machines}
\subsubsection{Rapid7 Nexpose}
De Rapid7 Nexpose is een Windows Server 2019 datacenter virtuele machine dat in het bachelorproef.com domein werd toegevoegd. Op deze virtuele machine draait de Rapid Nexpose software genaamd InsideVM. Rapid Nexpose is een oplossing voor het kwetsbaarheidsbeheer die risico's analyseert over kwetsbaarheden, configuraties en controles. Gebruikers kunnen op hun manier kwetsbaarheden in besturingssystemen, software van derden, webapplicaties, browsers en databases efficiënt beheren en problemen met verkeerde configuratie identificeren. De Rapid7 Nexpose virtuele machine werd specifiek gebruikt in combinatie met de Cisco Identity Services Engine virtuele machine voor het Thread-centric network access control use case. Op figuur \ref{fig:nexpose} is de home pagina van InsideVM weergegeven die u een visueel beeld geeft van de Rapid7 Nexpose software.
\newline\newline
Daarnaast werd de virtuele machine voorzien van de volgende specificaties: 
\begin{itemize}
	\item Geheugen: 48 GB Random Access Memory
	\item Opslag: 356 GB
	\item CPU: 24 cores
	\item Netwerk adapter: Cisco\textunderscore netwerk
\end{itemize}

Meer informatie over de installatie van Rapid7 Nexpose kan men terug vinden op de website van Rapid7(\cite{rapid7}).

\begin{figure}[H]
	\centering
	\includegraphics[width=0.9\textwidth]{nexpose.png}
	\caption{Home pagina van InsideVM}%
	\label{fig:nexpose}%
\end{figure}

\subsubsection{Cisco Identity Services Engine}
De tweede virtuele machine waarop dieper wordt op ingegaan is Cisco Identity Services Engine. Deze virtuele machine zal zich verdiepen in de use cases: 'Port-based network access control', 'Policy-based network access control' en 'Thread-centric network access control'. Daarvoor zijn een aantal configuraties uitgevoerd op de virtuele machine van Cisco Identity Services Engine om een antwoord te geven het onderzoek. Hieronder worden de configuraties opgesplitst per use cases, waarbij de use cases vaak in de praktijk worden gecombineerd. De resultaten van de Cisco Identity Services Engine configuraties vindt men terug in Hoofdstuk \ref{ch:Resultaten}.
\newline
\newline
Alvorens men dieper ingaat op de use cases werd Cisco Identity Services Engine toegevoegd aan het Active Directory Domain. Figuren \ref{fig:AD_Cisco1} en \ref{fig:AD_Cisco2} tonen aan dat de virtuele machine van Cisco Identity Services Engine probleemloos toegevoegd is aan het domein 'Bachelorproef.com'. Dit toont aan dat Cisco Identity Services Engine eenvoudig geïntegreerd kan worden met een domein waarbij extra policy rules  in de toekomst kunnen ingevoerd worden. Dit onderzoek focust zich op de bescherming van de beschikbare interfaces op de Cisco switch. Dit door een extra authenticatie methode te voorzien en dus niet door een integratie van 'Port-based network access control' met het gebruik van Active Directory Domain users.

Verder werd systeem geïnstalleerd op een Red Hat Enterprise 7 distributie, waar volgende specificaties voor werd vrijgemaakt: 

\begin{itemize}
	\item Geheugen: 128 GB Random Access Memory
	\item Opslag: 512 GB
	\item CPU: 24 cores
	\item Netwerk adapter: Cisco\textunderscore netwerk
\end{itemize}

De installatie van Cisco Identity Services Engine werd mogelijk door \cite{CiscoISE_InstallationGuide}.

\begin{figure}[H]
	\centering
	\includegraphics[width=0.9\textwidth]{AD_ise.png}
	\caption{Cisco ISE gelinkt met het AD domein}%
	\label{fig:AD_Cisco1}%
\end{figure}

\begin{figure}[H]
	\centering
	\includegraphics[width=0.7\textwidth]{ADComputers.png}
	\caption{Active Directory Users en Computers}
	\label{fig:AD_Cisco2}
\end{figure}

\setlength{\parindent}{5ex}\fontsize{12}{20}\textbf{Port- en Policy-based network access control }
 \newline
	Eerder werd vernoemd in sectie \ref{sec:config} dat er implementatie van 'Port-based network access control' configuraties op de Cisco switch verplicht zijn. Op dit onderdeel van de sectie ligt de focus op de configuratie van de 'Port-based network access control' use case in de Cisco Identity Services Engine webbrowser. 
	\newline
	\newline
	De Configuratie wordt gestart met de creatie van een aantal Cisco Identity Services Engine users in het 'Identity Management' tablad. In figuur \ref{fig:users} ziet men twee aangemaakte users waarbij user 'TestV2' zich in de groep 'Employees' bevindt en user 'Test' niet. Dit zal zich uiten wanneer een gebruiker zich aanmeldt met user 'Test'. Hierbij zal hij geen toegang krijgen tot het netwerk aan de hand van een policy rule.
 	\begin{figure}[H]
 		\centering
 		\includegraphics[width=0.9\textwidth]{Port_users.png}
 		\caption{Cisco Identity Services Engine users}%
 		\label{fig:users}%
 	\end{figure}
 Vervolgens werd in het tablad 'Network Resources' de Cisco switch vastgelegd waardoor Cisco Identity Services Engine communicatie met het netwerk apparaat kan vaststellen. Figuur \ref{fig:ISESwitch} toont de configuratie voor deze communicatie. 
 	
 	 	\begin{figure}[H]
 		\centering
 		\includegraphics[width=0.7\textwidth]{ISESwitch.png}
 		\caption{Configuratie van de Cisco switch in ISE}%
 		\label{fig:ISESwitch}%
 		\end{figure}
 	
 	Ten slotte is een policy rule ingesteld waarbij gecontroleerd wordt of de gegevens overeenkomen met de instellingen van de policy rule. In dit geval laat de policy rule netwerk trafiek toe als de gebruiker de juiste identificatie gegevens ingeeft maar anderzijds moet de gebruiker ook tot de groep 'Employees' toegewezen zijn. Op figuur \ref{fig:ISESwitch} is policy role weergegeven.
 	
	 	 \begin{figure}[H]
		\centering
		\includegraphics[width=0.7\textwidth]{PolicySet_Port.png}
		\caption{Configuratie van de 802.1x policy rule in ISE}%
		\label{fig:ISESwitch}%
		\end{figure}
	
	Om de Port-based en Policy-based network access control use case uit te breiden werd een 'Authorization Policy - Global Exceptions' ingevoerd zodat toegang op het netwerk met een Ethernet kabel vanuit de Cisco switch enkel mogelijk is op bepaalde weekdagen. Zo kunnen bedrijven een policy rule implementeren die de toegang naar het netwerk op weekend dagen blokkeert. Op figuur \ref{fig:weekend} is de Global Exceptions policy rule terug te vinden waarbij men de policy rule test in Hoofdstuk \ref{ch:Resultaten}. 
	
	\begin{figure}[H]
		\centering
		\includegraphics[width=0.7\textwidth]{Disable_At_Weekend_policy.png}
		\caption{Configuratie van de 802.1x weekend policy rule in ISE}%
		\label{fig:weekend}%
	\end{figure}
\setlength{\parindent}{5ex}\fontsize{12}{20}\textbf{Thread-centric en Policy-based network access control }
 \newline
 \newline
 In dit onderdeel van Cisco Identity Services Engine ligt de focus op de 'Thread-centric en Policy-based network access control' use case. Zoals voordien vermeld zijn de laatstgenoemde use cases gecombineerd omdat de Thread-centric network access control use case gebruikmaakt van policy rules. Hiervoor werd \cite{Rapid7_nexpose} als Thrid Party Vendor voor Tread-centric network access control gebruikt.
 \newline
 \newline
 Om de configuratie van Thread-centric network access control te starten wordt de 'Thread-centric NAC service' ingeschakeld. Op figuur \ref{fig:serviceThread} is te zien dat de service in \textbf{Administration > System > Deployment} werd ingeschakeld.
 
\begin{figure}[H]
		 	\centering
		 	\includegraphics[width=0.7\textwidth]{threadcentricService.png}
		 	\caption{Thread-centric NAC service in ISE}%
		 	\label{fig:serviceThread}%
\end{figure}

 Vervolgens werden de certificaten tussen Cisco Identity Services Engine, Windows Active Directory en Rapid7 Nexpose geconfigureerd. Indien men meer informatie wenst over de configuratie van deze certificaten kan men \cite{thread_yt} bekijken.
 \newline
 \newline
 Als volgt zijn er aantal instellingen uitgevoerd op de Rapid7 Nexpose applicatie zoals de creatie van een site die de assets bevat. In dit geval werd een laptop die reeds verbonden was met het afgezonderd netwerk gebruikt als asset. Op het eind van de configuratie zal Cisco Identity Service Engine een automatisch scan starten wanneer de laptop zich aansluit op het netwerk. 
 \newline
 \newline
 Communicatie tussen Cisco Identity Services Engine en Rapid7 Nexpose werd mogelijk door een 'Vendor Instance' aan te maken. Op figuur \ref{fig:vendor} is de configuraties van de 'Vendor Instance' weergegeven.
	 \begin{figure}[H]
		 	\centering
		 	\includegraphics[width=0.7\textwidth]{VendorRapid7.png}
		 	\caption{Rapid7 Vendor Instance in ISE}%
		 	\label{fig:vendor}%
	 \end{figure}
Nadien werden er Authorization Profiles aangemaakt die een Rapid7 Nexpose scan triggeren. Dit werd mogelijk door te navigeren naar \textbf{ Policy > Policy Elements > Results > Authorization > Authorization Profiles}. Er werden twee Authorization Profiles aangemaakt namelijk: 'WIRED\textunderscore VA\textunderscore USER' en 'WIRED\textunderscore VA\textunderscore QUARANTINE'. 
 \newline
\newline
Wanneer een eind apparaat verbinding maakt met het netwerk langs de Cisco switch dan wordt het eind apparaat via 'WIRED\textunderscore VA\textunderscore USER' toegewezen. Vervolgens wordt een automatisch scan gestart in Rapid7 Nexpose die getriggerd wordt door Cisco Identity Services Engine. 
 \newline
\newline
Indien het eind apparaat een bepaalde 'Nexpose-CVSS\textunderscore Base\textunderscore Score' heeft dan zal afhankelijk van de score het eind apparaat het 'WIRED\textunderscore VA\textunderscore QUARANTINE' Authorization Profiles toegewezen krijgen. Dit resulteert in een eind apparaat dat geen toegang meer heeft tot het netwerk. Figuur \ref{fig:WIRED} toont één configuratie van de Authorization Profiles namelijk de WIRED\textunderscore VA\textunderscore USER'. 
\newline
\newline
Belangijk om te weten is dat de 'WIRED\textunderscore VA\textunderscore USER' Authorization Profile de volgende 'Attributes Details' kent: 
	\begin{itemize}
	\item Access Type = ACCESS\textunderscore ACCEPT
	\item on-demand-scan-interval = 48
	\item periodic-scan-enabled = 24	
	\item va-adapter-instance = (Instance-id van de 'Adapter Instance')
\end{itemize}

 \begin{figure}[H]
	 	\centering
	 	\includegraphics[width=0.7\textwidth]{WIRED7.png}
	 	\caption{Rapid7 Vendor Instance in ISE}%
	 	\label{fig:WIRED}%
 \end{figure}

Tot slot zijn er nog de policy rules die eind apparaten toelaten of in
quarantaine plaatsen. Op figuur \ref{fig:quarin} ziet men de policy rules die geïmplementeerd zijn voor het Thread-centric network access control use case. Hierbij werd de Dot1x policy rule aangepast zodanig dat de Profiles Results gebruikmaakt van 'WIRED\textunderscore VA\textunderscore USER'. Daarnaast werd er een 'Authorization Policy - Global Exceptions' ingevoerd die eind apparaten in quarantaine plaatsen indien het rapport van Rapid7 Nexpose uitwijst dat het eind apparaat een Nexpose-CVSS\textunderscore Base\textunderscore Score heeft groter dan 4. 

  \begin{figure}[H]
 	\centering
 	\includegraphics[width=0.7\textwidth]{PolicyRuleThreadCentric.png}
 	\caption{PolicyrRules Thread-centric NAC in ISE}%
 	\label{fig:quarin}%
 \end{figure}
 
 De installatie van Thread-centric network access control in combinatie met Rapid7 Nexpose werd mogelijk door \cite{thread_yt}.

 


\subsubsection{Windows server 2019 datacenter}
Het gebruik van een Windows Server 2019 maakt de creatie van een Active Directory domein mogelijk. Een nieuwe forest werd hiervoor aangemaakt en de Windows Server 2019 werd meteen ook gepromoveerd naar de domeincontroller binnen het 'bachelrorproef.com' netwerk.
\newline
\newline
Via Cisco Identity Service Engine kan men subset van domeinen selecteren vanuit de vertrouwde domeinen voor authenticatie en autorisatie. Deze subset van domeinen worden authenticatiedomeinen genoemd. Het definiëren van deze authenticatiedomeinen verbetert de beveiliging voor bepaalde domeinen te blokkeren, waardoor de authenticatie van de gebruikers op deze domeinen wordt beperkt.
\newline
\newline
Zoals voorheen vermeld, is integratie met Windows Server Active Directory en Cisco Identity Services Engine reeds mogelijk. Hierdoor kunnen bedrijven met Cisco Identity Services opteren om werknemers te laten inloggen op het netwerk met de 'Domain Users’. Vaak wordt deze methode geïntegreerd met het '802.1x wireless port-based network access control' van Cisco Identity Services Engine om aan te melden op netwerk met gebruik met de Active Domain Directory gegevens. Helaas beschikte dit afgezonderde netwerk niet over een wireless controller die dit mogelijk maakt. Cisco Identity Services voorziet hiervoor vooraf gedefinieerde regels.

Verder wordt het ‘Remote Desktop Protocol (RDP)’ in combinatie met het Internet Protocol (IP) adres gebruikt om de Windows machines te bereiken.
\newline
\newline
Om de Windows server 2019 datacenter te gebruiken werden volgende specificaties vrijgemaakt:

\begin{itemize}
	\item Geheugen: 16 GB Random Access Memory
	\item Opslag: 126 GB
	\item CPU: 20 cores
	\item Netwerk adapter: Cisco\textunderscore netwerk
\end{itemize}

Als laatstse werd de installatie van Windows Server 2019 datacenter mogelijk door \cite{Win19_InstallationGuide}. 

\subsubsection{Pfsense}
De Pfsense virtuele machine verwijst naar de virtuele router waarbij één interface van de Pfsense gebruikt wordt als gateway voor alle vlan’s en van alle componenten die zich achter de Cisco switch bevinden. Vervolgens wordt de WAN interface van de Pfsense gebruikt om de data van en naar de Local Area Network interface te sturen. Hierdoor is er communicatie vanuit het thuisnetwerk met het afgezonderd netwerk mogelijk.
\newline
\newline
De pfsense is enkel te contacteren binnen het afgezonderd netwerk door ‘Hypertext Transfer Protocol (HTTP)’ te gebruiken. Met andere woorden kunnen enkel machines in het netwerk '192.168.17.0/24' de Pfsense interface bereiken door te surfen naar het ip-adres: '192.168.17.1'. Op figuur \ref{fig:Pfsense} is te zien dat de home interface van de Pfsense te bereiken is via 'http://192.168.17.1/'.
\newline
\newline
Ten slotte werd dit systeem geïnstalleerd op een Free Berkeley Software Distribution, waar volgende specificaties zijn voor vrijgemaakt:  

\begin{itemize}
	\item Geheugen: 4 GB Random Access Memory
	\item Opslag: 25 GB
	\item CPU: 4 cores
	\item Netwerk adapter: Cisco\textunderscore netwerk
\end{itemize}
De installatie van Pfsense werd mogelijk gemaakt door \cite{Pfsense_InstallationGuide}.

\begin{figure}[H]
	\centering
	\includegraphics[width=0.7\textwidth]{PfsenseHome.png}
	\caption{Home interface Pfsense}
	\label{fig:Pfsense}
\end{figure}

\subsubsection{Jumphost}
Een jumphost is een tussenliggende host machine waarbij connectie naar een extern netwerk mogelijk is. Vervolgens kan een verbinding worden gemaakt met een andere host in het extern netwerk. Met andere woorden 'jumpt' men dus van het ene apparaat naar het andere om een extern netwerk te bereiken. 
\newline
\newline
Dit werd in de Cisco Identity Services Engine omgeving ook gebruikt waarbij een eind apparaat in het thuisnetwerk verbinding maakt met de Jumphost via Remote Desktop Protocol om zo eind apparaten in het afgezonderd netwerk te contacteren. Hierbij zijn volgende instellingen gebruikt:  
 
 
\begin{itemize}
	\item Geheugen: 16 GB Random Access Memory
	\item Opslag: 126 GB
	\item CPU: 20 cores
	\item Netwerk adapter 1: Cisco\textunderscore netwerk
	\item Netwerk adapter 2: Sub\textunderscore netwerk
\end{itemize}

De installatie van de Jumphost werd mogelijk door \cite{Win19_InstallationGuide}. 

\subsection{Begrippen}
\textbf{Telenet Access Point} is een draadloze server die gegevens verzendt via radio golven in middel van een antenne. Deze gegevens worden ontvangen via het bedrage UTP netwerk waarbij deze “server” is op aan gesloten.
\newline
\newline
\textbf{Telenet Modem} is een toestel die informatie van uw internetprovider ontvangt via een telefoonlijn, glasvezelkabel of coaxkabel in de woning en converteert dit naar een digitaal signaal.
\newline
\newline
\textbf{Coax kabel} is een twee polige kabel die instaat voor het overdragen van beeld en geluid. Deze signalen treden gemakkelijker binnen, waarbij het signaal op korte afstanden snel zwak wordt. 
\newline
\newline
\textbf{Virtual local area network} is een type virtueel network dat gerealiseerd wordt op de datalinklaag. Dit bestaat uit een groep eindstations en switches die logisch gezien één enkel gemeenschappelijk local area network (LAN) vormen.
\newline
\newline
\textbf{Local Area Network} is een computernetwerk dat een relatief klein gebied beslaat. Meestal is een LAN beperkt tot een kamer, een gebouw of een groep gebouwen, waarbij een LAN kan via telefoonlijnen en radiogolven over elke afstand met andere LAN's kan communiceren.
\newline
\newline
\textbf{Wide Area Network} is een netwerk van apparaten, een verzameling van local area networks, die via draadloze of bekabelde communicatielijnen met elkaar verbonden zijn.

\section{Cisco Identity Services Engine enquête}
\label{sec:enquête}
Om een link te leggen met de resultaten van de uitgevoerde testen, werd een enquête of een formulier opstelt. Deze enquête is als onderliggende basis gebruikt bij de analyse van de resultaten die verkregen werden door de fysieke testen. Zoals in vorige hoofdstukken vermeld, is deze enquête een opiniepeiling. Waarbij een aantal vragen gesteld zijn aan personen die reeds gekend zijn met Cisco Identity Services Engine. Hierdoor is de doelgroep van deze enquête beperkt tot de Cisco Identity Services Engine specialisten.
\newline
\newline
Deze enquête is publiekelijk gemaakt via LinkedIn en via E-mail. Linkedin is een online sociaal netwerk dat is opgericht voor Vakmensen, die ongeveer 610 miljoen geregistreerden telt. Omdat deze enquête publiekelijk werd gemaakt op \cite{LinkedIn}, bevat dit formulier een aantal ‘failsave’ vragen. Deze ‘failsave’ vragen zijn bedoelt wanneer niet Cisco Identity Services Engine specialisten de enquête proberen in te vullen. Met als gevolg dat de kans op onjuiste ingevuld antwoorden verkleind werd.
\newline
\newline
Dit formulier of enquête werd mogelijk gemaakt door Microsoft en Hogeschool Gent. Elke student heeft recht op een gelicentieerd Office 365 pakket gedurende zijn opleidingstraject. Het programma in kwestie noemt men \cite{MicrosoftForms} dat inbegrepen is in het Microsoft Office 365 pakket.
\newline
\newline
Een overzicht van de vragen die gesteld werden tijdens de enquête is terug te vinden in \ref{ch:Resultaten_enquête}. Bij elke vraag is steeds een klein woordje uitleg gegeven. Idem zoals de mogelijke antwoorden en eventuele doorverwijzingen naar andere vragen.




 \chapter{\IfLanguageName{dutch}{Resultaten}{Resultaten}}
\label{ch:Resultaten}
In dit hoofdstuk zullen de resultaten van de testen en de enquête verwerkt worden om tot een antwoord te komen van dit onderzoek. Hierbij zijn de resultaten van de Cisco Identity Services Engine omgeving opgedeeld in 'Port-based en Policy-based network access control' en 'Thread-Centric en Policy-based network access control '. Vervolgens zijn de enquête resultaten verwerkt met een aantal grafieken waarbij de resultaten terug te vinden zijn in Bijlage \ref{ch:Resultaten_enquête}. Meer informatie over de resultaten van de enquête vindt men in sectie \ref{sec:enqueteISe}.

\section{Cisco Identity Services Engine omgeving}
Zoals men ziet wordt Policy-based network access control steeds gecombineerd met Port-based en Thead-Centric network access control. Dit heeft als reden dat zowel Port-based, als voor Thread-Centric network access control, policy rules zijn toegepast. In elk van deze subsecties worden de resultaten van voor de implemenatie van de Cisco Identity Services Engine use cases  en na de implemenatie van deze use case toegelicht. Op die manier kan in hoofdstuk \ref{ch:conclusie} een conclusie gemaakt worden die een duidelijk antwoord biedt op dit onderzoek.  
\subsection{Port-based en Policy-based network access control}
Wanneer de use cases 'Port-based en Policy-based network access control' niet worden toegepast, dan kunnen gebruikers zich aansluiten op het netwerk zonder enige beveiliging. Dit is natuurlijk enkel mogelijk wanneer alle interfaces op het netwerk apparaat ingesteld zijn met het correcte vlan id. Anderzijds is het duidelijk dat eind apparaten zich eenvoudiger kunnen aansluiten op het netwerk zonder de implementatie van 'Port-based en Policy-based network access control' use cases.
Als men de resultaten van de implemenatie van 'Port-based en Policy-based network access control' erbij halen, dan ziet men een duidelijk verschil in beveiliging ten opzichte van een netwerk zonder de use cases. Dit verschil is duidelijk te merken wanneer een gebruiker zich probeert aan te melden op het netwerk via één van de interfaces op de fysieke Cisco switch. Het eind apparaat wijst zichzelf een ip-adres toe van de vorm 169.254.x.x dat geen enkel compartiment van het netwerk kan bereiken. 
\newline
\newline
Het is echter wanneer de 'Wired AutoConfig' services aanstaan dat men zich kan aanmelden op het netwerk door de volgende stappen uit te voeren: 
\begin{itemize}
	\item Open het configuratiescherm, en ga naar 'netwerk en internet'.
	\item Open vervolgens het 'netwerkcentrum'.
	\item Open 'Adapter instellingen wijzigen'.
	\item Open het 'properties' tablad, door een rechtermuisklik op de correcte "Ehternet adapter".
	\item Vervolgens verschijnt het tablad "Authenticatie".
	\item Open "Extra instellingen".
	\item Veranderd de 'authentication modus' naar 'gebruikers authenticatie'.
\end{itemize}

Als de eind gebruikers de juiste gebruikersnaam en wachtwoord ingeeft, wordt de gebruiker met zijn apparaat aangesloten op het netwerk. Belangrijk om te weten is dat er een policy rule zo ingesteld is dat alleen gebruikers van de groep 'Employees' toegang krijgen tot het netwerk. Het gebruik van de policy werd al eerder vernoemd in het hoofdstuk \ref{ch:Proof of concept}.
Figuur \ref{fig:Test_gebruiker} toont aan dat wanneer men wenst in te loggen met de gebruikersnaam 'Test' geen toegang krijgt tot het netwerk. Gebruiker 'Test' bevindt zich niet in de groep 'Employees' en zal dus als gevolg geen toegang krijgen tot het netwerk. Dankzij het gebruik van policy rules wordt de proef aangetoond dat policy rules ook de vruchten kan plukken op beveiliging.

\begin{figure}[H]
	\centering
	\subfloat{{\includegraphics[width=5.5cm]{Test_no_employee.png} }}%
	\qquad
	\subfloat{{\includegraphics[width=5.5cm]{Test_no_successed.png} }}%
	\newline
	\subfloat{{\includegraphics[width=11.5cm]{TestDenied.png} }}%
	\caption{Proef resultaat met een niet employee gebruiker}%
	\label{fig:Test_gebruiker}%
\end{figure}

Als men inlogt met een gebruiker die in de Cisco Identity Services Engine groep 'Employees' bevindt dan zal de gebruiker zonder problemen zich kunnen aansluiten op het netwerk. Dit wordt aangetoond in figuur \ref{fig:Test_gebruiker} waarbij een proef met de gebruiker 'TestV2' wordt uitgevoerd. Gebruiker 'TestV2' bevindt zich in groep 'Employees' waardoor het vanzelfsprekend is dat 'TestV2' met zijn eind apparaat met het netwerk kan verbinden. 

\begin{figure}[H]
	\centering
	\subfloat{{\includegraphics[width=6cm]{TestV2_Employee.png} }}%
	\qquad
	\subfloat{{\includegraphics[width=6cm]{TestV2_succeeded.png} }}%
	\newline
	\qquad
	\subfloat{{\includegraphics[width=12cm]{TestV2_succeeded_ISE.png} }}%
	\caption{Proef resultaat met een employee gebruiker}%
	\label{fig:Test_gebruiker}%
\end{figure}

Vervolgens werd bij de tweede proef een policy rule geïmplementeerd die de toegang naar het netwerk vanuit de Cisco switch blokkeert op specifieke weekdagen.  De proef zou moeten aantonen wanneer gebruiker 'X' zich op woensdag aansluit op het netwerk dan zou gebruiker 'X' met zijn eindapparaat geen toegang krijgen.
Op Figuur \ref{fig:Woensdag} is te zien dat gebruiker 'TestV4' met de 'Port-based network access control' use case zich probeert aan te sluiten op het netwerk. 

\begin{figure}[H]
	\centering
	\includegraphics[height=0.20\textheight]{TestV4_Employee.png}
	\caption{Grafiek resultaat vraag 2}
	\label{fig:Woensdag}
\end{figure}

Het is al snel duidelijk dat Cisco Identity Services Engine toegang tot het netwerk blokkeert. Cisco Identity Services Engine registreert dat gebruiker 'TestV4' zich probeert aan te sluiten op het netwerk maar wordt vervolgens de toegang onmiddelijk ontnomen. Dit wordt aangetoond op figuur \ref{fig:failed} waarbij de toegang tot het netwerk voor gebruiker 'TestV4' geblokkeerd werd. 

\begin{figure}[H]
	\centering
	\includegraphics[height=0.20\textheight]{UsersFailed_Suc.png}
	\caption{Grafiek resultaat vraag 2}
	\label{fig:failed}
\end{figure}

Om aan te tonen dat gebruiker 'TestV4' het netwerk werd ontnomen door de 'Disable\textunderscore at\textunderscore weekend' policy rule kan men het rapport terug vinden in figuur \ref{fig:failed2}. Op deze figuur is te zien dat 'Disable\textunderscore at\textunderscore weekend' het desbetreffende Authorization policy rule was die werd uitgevoerd dat resulteerde in geen toegang tot het netwerk. 

\begin{figure}[H]
	\centering
	\includegraphics[height=0.20\textheight]{TestV4_failed.png}
	\caption{Grafiek resultaat vraag 2}
	\label{fig:failed2}
\end{figure}

Ten slotte toont figuur \ref{fig:ping} aan dat het eindapparaat na blokkade van het netwerk geen enkel component in het afgezonderd netwerk kan bereiken. Hiervoor werd het commando: 'Ping 192.168.17.1' uitgevoerd waarbij het eindapparaat de default gateway tracht te bereiken.

\begin{figure}[H]
	\centering
	\includegraphics[height=0.20\textheight]{PingTestV4_failed.png}
	\caption{Grafiek resultaat vraag 2}
	\label{fig:ping}
\end{figure}

In de sectie \ref{sec:trepo} worden de resultaten van Tread-Centric en Policy-based network access control besproken. Voor meer uitleg over de volledige conclusie kan men terecht in Hoofdstuk \ref{ch:conclusie}.


\subsection{Thread-Centric en Policy-based network access control}
\label{sec:trepo}
In dit onderdeel van het onderzoek ligt de focus op de resultaten van de 'Thread-centric en policy-based network access control' use cases. De resultaten zouden moeten uitwijzen wanneer het eindapparaat een kwetsbaarheid niveau van 4 of meer behaald zou Cisco Identity Services Engine dit eindapparaat in quarantaine moeten plaatsen.
\newline
\newline
Figuur \ref{fig:api} indiceert dat gebruiker 'TestV5' met zijn eindapparaat zonder enige problemen kan aansluiten op het netwerk dat gelinkt werd met het 'WIRED\textunderscore VA\textunderscore USER' Authorization Profile. Na enkele minuten ontvangt Cisco Identity Services Engine het rapport van Rapid7 Nexpose en merkt dat het eindapparaat een Nexpose-CVSS\textunderscore Base\textunderscore Score heeft van 6.2. Dit heeft een hogere score dan in de policy rule werd voorzien. Hierbij plaatst Cisco Identity Services Engine het eindapparaat in de 'WIRED\textunderscore VA\textunderscore QUARANTINE' Authorization Profile dat geen toegang heeft tot het netwerk. 

\begin{figure}[H]
	\centering
	\includegraphics[height=0.20\textheight]{ThreadSucceeded.png}
	\caption{Grafiek resultaat vraag 2}
	\label{fig:api}
\end{figure}

Figuur \ref{fig:rapid} demonstreert het Rapid7 Nexpose scan rapport van asset '192.168.17.6' dat getriggerd werd door Cisco Identity Services Engine. Hierbij is een kwetsbaarheid weergegeven die een Base\textunderscore Score heeft van 6.2.
\begin{figure}[H]
	\centering
	\includegraphics[height=0.20\textheight]{InsightVM_base.png}
	\caption{Grafiek resultaat vraag 2}
	\label{fig:rapid}
\end{figure}
Vervolgens kan men een Base\textunderscore Score van 6.2 terugvinden in het Authorization Profile rapport dat werd gegenereerd door Cisco Identity Services Engine. Zoals men ziet ligt dit volledig in lijn met de Base\textunderscore Score van het Rapid7 Nexpose scan rapport. Dit Authorization Profile rapport wordt weergegeven in figuur \ref{fig:detail}. 
\begin{figure}[H]
	\centering
	\subfloat{{\includegraphics[width=6cm]{thread_overview_va_Quar.png} }}%
	\qquad
	\subfloat{{\includegraphics[width=6cm]{thread_base.png} }}%
	\caption{Proef resultaat met een employee gebruiker}%
	\label{fig:detail}%
\end{figure}

Bij dezen werden de resultaten van 'Thread-centric en Policy-based network access control' besproken. De resultaten hiervan zijn verwerkt in Hoofstuk \ref{ch:conclusie}.
\section{Cisco Identity Services Engine enquête}
\label{sec:enqueteISe}
In deze sectie worden de resultaten van de enquête globaal besproken. Indien men meer informatie wenst over de resultaten per vraag kan men Bijlage \ref{ch:Resultaten_enquête} raadplegen.
\newline
\newline
Bovenal mag men zeker tevreden zijn met de responsen van de enquête daarnevens kent de enquête 14 responsen op 2 weken tijd. Helaas beschikt men niet over het grote netwerk om de responsen van de enquête te doen verhogen. Een groter aantal responsen zou resulteren in een veel nauwkeuriger resultaat.  
\newline
\newline
78.6\% van de geënquêteerden is gekend met het Cisco Identity Services Engine product. De overige 21.4\% is niet gekend met het Cisco Identity Services Engine product en is dus niet vertrouwd met network access control technolgieën. Bij gevolg weten de geënquêteerden niet welk alternatief in hun omgeving wordt toegepast waardoor vraag 4 en 5 geen antwooren kent. Van de personen die gekend zijn met Cisco Identity Services Engine is bij iedereen het product geïmplementeerd in hun bedrijfsomgeving. Wanneer de vraag werd gesteld waar ze voor het eerst Cisco Identity Services Engine hebben gehoord. Dan reageert 71.4\% van de responsen met het antwoord 'In de organisatie'. Het overschot van de responsen is gelijk verdeelt tussen 'Tijdens een webinair', ‘Tijdens een opleiding' en 'Op het internet'. Figuur \ref{fig:vraag2} kan hiervoor geraadpleegd worden. 
\begin{figure}[H]
	\centering
	\includegraphics[height=0.20\textheight]{Vraag2.png}
	\caption{Grafiek resultaat vraag 2}
	\label{fig:vraag2}
\end{figure}
De resultaten waarbij men de rede waarom men Cisco Identity Services Engine heeft gekozen in een hun bedrijfsomgeving moest ingeven, ligt gelijkmatig verspreid. Wat wel duidelijk is dat het antwoord 'Omwille van de betere beheerbaarheid van eind apparaten.' een hoger percentage kent dan de overige antwoorden. Dit ziet met ook in grafiek \ref{fig:graf6} duidelijk terug.
\begin{figure}[H]
	\centering
	\includegraphics[width=0.50\textwidth]{Vraag6.png}
	\caption{Grafiek resultaat vraag 6}
	\label{fig:graf6}
\end{figure}
Uit vraag 7 kan afgeleidt worden dat 'Identiteits- en toegangsbeer','beleidshandhaving', 'Functionaliteit' en 'Uitbreidbaarheid' de belangrijkste kermerken zijn voor de keuze van het type network access control. Figuur \ref{fig:vraag7} toont daarbij de resultaten terug waarbij zeer duidelijk te zien is dat de vorige benoemde resultaten een zeer belangrijke rol spelen in de keuze van het network access control product. 

\begin{figure}[H]
	\centering
	\includegraphics[height=0.30\textheight]{Vraag7.png}
	\caption{Grafiek resultaat vraag 7}
	\label{fig:vraag7}
\end{figure}

Voor de volgende vraag kan men concluderen dat de veranderingen die bedrijven ondervinden na implementatie van Cisco Identity Services Engine verschillende zijn van responder tot responder. Globaal gezien komt dit wel overeen met elkaar dat Cisco Identity Services Engine de zichtbaar van eind apparaten op het netwerk verhogen. Een verbetering in troubleshooting is ook een van de veranderingen die de geënquêteerden ondervonden. Meer informatie over deze antwoorden is terug te vinden in bijlage \ref{tab:vraag8}. Vervolgens vindt 54.5\% van de geënquêteerden dat Cisco Identity Services Engine het product gelijkaardig vindt ten opzichte van andere network access control. Het is echter zo dat 9.1\% van de responders vindt dat Cisco Identity Services Engine slechter is dan zijn concurrenten. De rede achter dit kan helaas niet onbeantwoord worden. Verder bevat 36.4\% van de antwoorden het antwoord 'Beter'. Figuur \ref{fig:vraag9} geeft deze verwoording visueel terug. 

\begin{figure}[H]
	\centering
	\includegraphics[height=0.30\textheight]{Vraag9.png}
	\caption{Grafiek resultaat vraag 9}
	\label{fig:vraag9}
\end{figure}

De resultaten van de vraag: 'Wat zijn volgens u de meeste voorkomende modules of use cases van Cisco Identity Services Engine?', zijn in figuur \label{fig:vraag10} terug te vinden. Hierbij is duidelijk te zien dat merendeel van de geënquêteerden vindt dat 'Identity-based network access control' de belangrijkste use case van Cisco Identity Services Engine. Vervolgens komt het 'Policy-based network access control' dat daarna gevolgd wordt door het 'Thread-Centric network access control'. Hieruit kan men concluderen dat de proef een onderzoek uitvoert naar de bijna belangrijkste use cases. Het is toch verbazend dat slecht 7.7\% van de personen koos voor een 'Port-based network access control' use case. 

\begin{figure}[H]
	\centering
	\includegraphics[height=0.30\textheight]{Vraag10.png}
	\caption{Grafiek resultaat vraag 10}
\end{figure}

Daarnaast blijkt ook dat 54.5\% zegt dat Cisco Identity Services Engine nadelen kent binnen een netwerk. Deze nadelen uiten zich als 'Certificate updates require nac to shut down', 'het netwerk is afhankelijk van de nac oplossing', 'Instabiliteit van connecties met sommige systemen', enzovoort. Verder werd er ook heel wat voordelen opgesomd door de vakspecalisten die weergegeven zijn in \ref{tab:vraag11}. Alle antwoorden zijn terug te vinden in bijlage \ref{tab:vraag13}.
\newline
\newline
Uit vragen 14 en 15 is gebleken dat Cisco Identity Services Engine een logische keuze is ten opzichte van andere network access control producten. De resultaten van vraag 14 is 90.9\% voor het antwoord 'Ja' en 9.1\% met het antwoord 'Neen'. Vraag 15 toont aan dat 100\% van de vakspecalisten vindt dat een network access control product noodzakelijk is in een bedrijfsomgeving. Hieruit kan men concluderen dat een network access control zoals Cisco Identity Services Engine voor velen een echte must is.
Hierbij zijn de Antwoordenterug te vinden in bijlage \ref{tab:vraag13}. Verder werd er ook heel wat voordelen opgesomd door de vakspecalisten die weergegeven zijn in \ref{tab:vraag11}.
\newline
\newline
18.2\% van de personen die de enquête invulden vinden dat er functionaliteiten ontbreken in Cisco Identity Services Engine. De betreffende missende functionaliteiten zijn 'Industriële gerichtheid, incl. goede support voor staticshe ip adressen.' en 'Saml integratie in een samenwerking met Azure AD en Captive portal voor byod.’.


Tot slot kreeg Cisco Identity Services Engine een gemiddelde beoordeling van 3.91 op 5 waarbij 90.9\% van de geënquêteerden dit product zou aanbevelen aan anderen.





%%=============================================================================
%% Conclusie
%%=============================================================================

\chapter{Conclusie}
\label{ch:conclusie}

In dit onderzoek wordt een antwoord gegeven op de onderzoeksvraag 'Hoe kunnen we een evoluerend bedrijfsnetwerk beter gaan beveiligen door het gebruik van Cisco Identity Services Engine?' Hiervoor is er een studie uitgevoerd waarbij verschillende proeven zijn uitgewerkt die de nood van de Cisco Identity Services Engine use cases naar boven haalt. Vervolgens is de studie aangevuld met een enquête die een opiniepeiling vormt over Cisco Identity Services Engine en zijn use cases.
\newline
\newline
Uit de resultaten van de 'Port-based en Policy-based network access control' use cases bekomt men een positieve resultaat. Hieruit bleek dat integratie van het desbetreffende use cases het bedrijfsnetwerk op bekabeld niveau robuster maakt. In het praktisch deel van het onderzoek is er toch een duidelijk verschil op te merken tussen een netwerk met 'Port-based en Policy-based network access control' en een netwerk zonder de use cases. Het is als crimineel veel eenvoudiger om verbinding te maken met het netwerk wanneer de use cases niet werden geïmplementeerd. Hierbij wordt geen bijkomend authenticatie proces gebruikt wanneer men connectie tracht te maken met het netwerk via een Ethernet kabel. Vervolgens betuigt de proef de implementatie van een policy rule waarbij netwerktoegang op bepaalde weekdagen wordt geblokkeerd een gunstig resultaat. Uit deze proef blijkt dat de policy rule een nuttig gegeven is die het netwerk verdedigd op weekdagen wanneer niemand aanwezig is in het bedrijf.

Uit de resultaten van de 'Thread-Centric en Policy-based network access control' use cases blijkt dat een netwerk wordt beschermt wanneer eindapparaten een bepaald kwetsbaarheid niveau bereiken. Dit resulteert zich in een netwerk waarbij eindapparaten die een kwetsbaarheid vormen in quarantaine worden geplaatst. De quarantaine ontneemt de toegang tot het netwerk van deze eindapparaten. Met als gevolg kunnen eindapparaten wanneer ze besmet zijn met malvare het netwerk niet verder infecteren. 
\newline
\newline
De snelheid waarbij een eindapparaat het netwerk verder infecteert is afhankelijk van de hardware dat wordt gebruikt voor Cisco Identity Sevices Engine en zijn Thrird Party Vendor. In de proef had Cisco Identity Services Engine enkele minuten nodig om het eindapparaat in quarantaine te plaatsen. Tijdens deze enkele minuten kan het eindapparaat het netwerk nog steeds infecteren. Hiervoor kan een toekomstig onderzoek worden uitgevoerd.


!! enquete 


Uit alle resultaten blijkt dat de uitkomst van de uitgevoerde proeven grotendeels in lijn liggen met de resultaten van de enquête. Hierbij biedt dit onderzoek zeker een meerwaarde voor bedrijven die hun eindapparaten beter willen beschermen en beheren tegen interne of externe threads. Het geeft een duidelijk beeld dat integratie van een network access control heel wat voordelen heeft in een bedrijfsnetwerk. Natuurlijk biedt dit onderzoek nog heel wat ruimte voor verder onderzoek. Een denkbare vervolgonderzoek kan een integratie van 'Identity-based network access control' use case in een bedrijfsnetwerk zijn.
'
%%=============================================================================
%% Bijlagen
%%=============================================================================

\appendix
\renewcommand{\chaptername}{Appendix}

%%---------- Onderzoeksvoorstel -----------------------------------------------

\chapter{Onderzoeksvoorstel}

Het onderwerp van deze bachelorproef is gebaseerd op een onderzoeksvoorstel dat vooraf werd beoordeeld door de promotor. Dat voorstel is opgenomen in deze bijlage.

% Verwijzing naar het bestand met de inhoud van het onderzoeksvoorstel
\input{../voorstel/voorstel-inhoud}

%%---------- Andere bijlagen --------------------------------------------------
% TODO: Voeg hier eventuele andere bijlagen toe
\input{Resultaten_enquête}

%%---------- Referentielijst --------------------------------------------------

\printbibliography[heading=bibintoc]

\end{document}
