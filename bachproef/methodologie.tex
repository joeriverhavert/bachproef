%%=============================================================================
%% Methodologie
%%=============================================================================

\chapter{\IfLanguageName{dutch}{Methodologie}{Methodology}}
\label{ch:methodologie}

%% TODO: Hoe ben je te werk gegaan? Verdeel je onderzoek in grote fasen, en
%% licht in elke fase toe welke stappen je gevolgd hebt. Verantwoord waarom je
%% op deze manier te werk gegaan bent. Je moet kunnen aantonen dat je de best
%% mogelijke manier toegepast hebt om een antwoord te vinden op de
%% onderzoeksvraag.

Deze bachelorproef voert een kwantitatief en kwalitatief onderzoek uit om een antwoord te vormen op de volgende onderzoeksvraag: "Een bedrijfsnetwerk beter beveiligen en beheren met behulp van Cisco Identity Services Engine". 
\newline
\newline
Vervolgens is een uitgebreide literatuurstudie of een stand van zaken ook te achterhalen in dit onderzoek. Deze literatuurstudie is terug te vinden in het hoofdstuk \ref{stand-van-zaken} van dit onderzoek. Implementaties van Cisco Identity Services Engine, met zijn bijhorende uses cases en zijn testen is in een thuisnetwerk afgenomen. Hiervoor is hardware voorzien die de installaties van Cisco Identity Service Engine en zijn use cases mogelijk maakt. Deze hardware is overhandigd door de Axians - Fit IT.
\newline
\newline
\section{Informatie verzameling}
De literatuurstudie of stand van zaken beschrijft een aantal belangrijke basisbegrippen en producten rondom de network access control technologie. Aansluitend zijn er een aantal voorgaande wetenschappelijke artikelen omtrent network access controls benaderd. Deze raadpleging van wetenschappelijke artikelen is aangeboden door Google Scholar, SpringerLink en de bibliotheek van Universiteit Gent.
\newline
\newline
Aan de hand van deze literatuurstudie is een enquête opgesteld, die een opiniepeiling vormt van de network access control technologiën waarbij Cisco Identity Services Engine centraal staat. De enquête bestaat uit dertien keuzevragen, vier open vragen en één 3-punts likertschaal. De opiniepeiling is te danken aan Microsoft. Zij bieden een Office 365 pakket, waarbij de Microsoft Forms applicatie standaard is inbegrepen. \newline \newline
Vervolgens is er verdere informatie verzameld met behulp van een aantal testen rondom Cisco Identity Services Engine. Deze beproevingen zijn uitgevoerd wanneer implementaties van Cisco Identity Service Engine, de drie Identity Services Engine modules en andere hardware componenten compleet zijn. In de hoop dat dit een beter beeld opleverd, die een antwoord zal bieden op de onderzoeksvraag van deze bachelorproef. 

\section{Onderzoeksverloop}
Dit onderzoek is verder gezet met het opzetten en configureren van de voorziene hardware, dat volgt na het schrijven van de volledige stand van zaken. De hardware verzameling bestaat uit een EXSi server, één Cisco switch met 10 Ethernet poorten en 2 SFP poorten, één bestaande laptop en een aantal opgestelde virtuele machines die de installaties en uitgevoerde testen mogelijk maakte. Hiervoor is een afgezonderd netwerk opgebouwd binnen het thuisnetwerk. 
Dit onderdeel van het onderzoek heet de implementatiefase, waarbij de installaties en implementaties van volgende use cases: “Policy-based access control” , “Threat-centric access control” en “Port-based access control” ook tot dit ontwikkelingsstadium behoorde. 
\newline
\newline
Het was van groot belang dat dit ontwikkelingsstadium van het onderzoek zorgvuldig werd uitgewerkt. Indien zich onafgewerkte zaken voordeden, dan kon zich dit uiten in een negatieve invloed op de resultaten van het onderzoek. 
Belangrijk om te vermelden is dat volgend stadium pas mogelijk was na alle implementaties en installaties rondom Cisco Identity Services Engine. Meer informatie over deze opstelling is terug te vinden in Hoofdstuk \ref{ch:Proof of concept}.
\newline
\newline
Vervolgens is er de enquête, fase drie van dit onderzoek. Het doel van deze enquête is enerzijds het creëren van een beter beeld omtrent de veiligheid en beheersbaarheid in een bedrijfsnetwerk met behulp van Cisco Identity Services Engine bij firma’s zoals de stagegever. Maar anderzijds is de enquête ook een opiniepeiling of de tevredenheidsbevraging over het Cisco Identity Services Engine product bij firma’s waar dit network access control product geïmplementeerd werd. Deze data is op het einde mee verwerkt met de resultaten van de uitgevoerde testen om zo een antwoord te geven op dit onderzoek. Meer informatie over de enquête is terug te vinden in Hoofdstuk \ref{ch:Proof of concept}.
\newline
\newline
Eens de bevraging voltooid werd, schakelde het onderzoek over naar de volgende fase. Deze fase heet de testingfase. Tijdens dit stadium van het onderzoek, is de implementatie van Cisco Identity Services Engine met zijn use cases uitgebreid getest. Met andere woorden werden de testen fysiek en met hardware zorgvuldig uitgevoerd. Het was van groot belang dat er goed werd nagedacht over de uitvoering van deze fysieke testen aangezien dit stadium het resultaat van het onderzoek heel snel kon beïnvloeden. Dit is gemakkelijk aangetoond met een voorbeeld. Stel dat de testen omtrent de “Port based access control” use case verkeerd werd geïmplementeerd. De test zou aantonen dat er geen verschil is op vlak van veiligheid en beheersbaarheid tussen een netwerk met of zonder een network access control product zoals Cisco Identity Services Engine. 
\newline
\newline
Het onderzoek is afgesloten met een data analyse. Deze fase van het onderzoek wordt pas ingevoerd wanneer alle data verzameld of ontvangen is. Deze ontvangen data werd vervolgens geanalyseerd om te verwerken in het besluit of conclusie van deze paper. Meer informatie omtrent de dataverzameling is terug te vinden in sectie \ref{sec:Dataverzameling}. 

\section{Data verzameling}
\label{sec:Dataverzameling}
Voor de interpretatie van het besluit van deze paper, is gebruik gemaakt van data die ontvangen werd uit de enquête die meer een opinie schept over het gebruik van network access control producten zoals Cisco Identity Services Engine. Maar anderzijds is er ook data verzameld door de geïmplementeerde uses cases zorgvuldig te testen.

Vervolgens is de verzameling van data gebruikt bij de data analyse. Data wordt analyseerd aan de hand van een statisch software programma, dat terug te vinden is in sectie \ref{sec:Dataanalyse}. Informatie van de testen is geanalyseerd aan de hand van de verkregen resultaten. 

De resultaten van de enquête zijn terug te vinden als bijlage.

\section{Data analyse}
\label{sec:Dataanalyse}
De resultaten van de data zijn verwerkt in een statisch software programma, genaamd \cite{RStudio}. RStudio is een geïntegreerde ontwikkelomgeving voor R, een programmeertaal voor statistische berekeningen en grafische afbeeldingen. Deze software is beschikbaar in twee soorten formaten:
\begin{itemize}
\item RStudio Desktop is een gewone desktop-applicatie. 
\item RStudio Server is een applicatie die draait op een externe server. Toegang tot dit RStudio formaat is mogelijk via de webbrowser.
\end{itemize} 
Voor dit onderzoek is 'RStudio Desktop' gebruikt. Hierbij hielp het RStudio programma bij de verzameling, invoering, lezen en bewerken van data of gegevens.
\begin{figure}[H]
	\centering
	\includegraphics[width=0.5\textwidth]{RStudio.png}
	\caption{Logo RStudio applicatie (\cite{RStudioLogo})}
	\label{fig:SPPS}
\end{figure}
De resultaten van de uitgevoerde testen, zijn in de mate van het mogelijke ook verwerkt in het RStudio programma. Dit is echter een heel klein onderdeel. De overige data die niet verwerkt is in het RStudio programma, is geanalyseerd op basis van de resultaten van de testen. Deze analyse is uitgevoerd door het opstellen van meerdere vergelijking. De analyse gebeurde met woorden door een vergelijking op te stellen met en eens zonder de integratie van de Cisco Identity Services Engine use cases. Bijvoorbeeld: Voor de use case 'Port-based network access control' is een test uitgevoerd waarbij de use case niet geïmplementeerd was. Dit uitte zich in het aankoppelen in het netwerk zondere enige beveiliging. Anderszijds is een test uitgevoerd waarbij de 'Port-based AC' geïmplementeerd werd, dat zich uitte in het niet automatisch koppelen aan het netwerk.   
\newline
\newline
Meer informatie over deze data is terug te vinden in het hoofdstuk \ref{ch:Proof of concept}.


\section{Validiteit en betrouwbaarheid}
Ten behoeve van de validiteit is de enquête opgesteld op basis van de literatuurstudie. Het gaat hierbij om literatuurstudie die geselecteerd is op relevantie voor de onderzoeksvraag. Daarnaast is enkel de meest recente literatuurstudie voor het onderzoek geraadpleegd.
\newline
\newline
Alvorens deze enquête werd uitgezet, is de enquête getest door een aantal personen binnenshuis. Hierbij lag de nadruk of de vragen duidelijk en goed geformuleerd waren. Daarnaast is ook gekeken hoe lang de invultijd van deze enquête bedroeg. Om de precisie van de data te verhogen is een likertschaal en zijn meerkeuzevragen gebruikt. Vervolgens is de enquête opgenomen door een aantal vakspecialisten, die de enquête konden raadplegen op LinkedIn. Er werd vertrouwelijk omgegaan met de gegevens door de resultaten geheel anoniem te verwerken. Hierdoor is het onderzoek valide.
\newline
\newline
Om de antwoorden van de geënqueteerde, te beproeven, zijn de resultaten vergeleken en geanalyseerd samen met de resultaten van de uitgevoerde testen. Helaas kan de betrouwbaarheid van deze testresultaten niet met 100 procent gewaarborgd worden, omdat tijdens de uitvoering van de testen van alles kan mis lopen. Maar aangezien er een analyse wordt opgesteld die de beide resultaten met elkaar gaat vergelijken, wordt de betrouwbaarheid zeker en vast aanzienlijk verhoogd.
\newline
\newline
Verder zijn de specifieke testen meermalig uitgevoerd in verschillende omstandigheden. Door gebruik te maken van veelvoudige testen, werd de herhaalbaarheid verhoogd. Hiermee is ook de validiteit van deze testen gewaarborgd.

