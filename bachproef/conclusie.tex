%%=============================================================================
%% Conclusie
%%=============================================================================

\chapter{Conclusie}
\label{ch:conclusie}

In dit onderzoek wordt een antwoord gegeven op de onderzoeksvraag 'Hoe kunnen we een evoluerend bedrijfsnetwerk beter gaan beveiligen door het gebruik van Cisco Identity Services Engine?' Hiervoor is er een studie uitgevoerd waarbij verschillende proeven zijn uitgewerkt die de nood van de Cisco Identity Services Engine use cases naar boven haalt. Vervolgens is de studie aangevuld met een enquête die een opiniepeiling vormt over Cisco Identity Services Engine en zijn use cases.
\newline
\newline
Uit de resultaten van de 'Port-based en Policy-based network access control' use cases bekomt men een positieve resultaat. Hieruit bleek dat integratie van het desbetreffende use cases het bedrijfsnetwerk op bekabeld niveau robuster maakt. In het praktisch deel van het onderzoek is er toch een duidelijk verschil op te merken tussen een netwerk met 'Port-based en Policy-based network access control' en een netwerk zonder de use cases. Het is als crimineel veel eenvoudiger om verbinding te maken met het netwerk wanneer de use cases niet werden geïmplementeerd. Hierbij wordt geen bijkomend authenticatie proces gebruikt wanneer men connectie tracht te maken met het netwerk via een Ethernet kabel. Vervolgens betuigt de proef de implementatie van een policy rule waarbij netwerktoegang op bepaalde weekdagen wordt geblokkeerd een gunstig resultaat. Uit deze proef blijkt dat de policy rule een nuttig gegeven is die het netwerk verdedigd op weekdagen wanneer niemand aanwezig is in het bedrijf.

Uit de resultaten van de 'Thread-Centric en Policy-based network access control' use cases blijkt dat een netwerk wordt beschermt wanneer eindapparaten een bepaald kwetsbaarheid niveau bereiken. Dit resulteert zich in een netwerk waarbij eindapparaten die een kwetsbaarheid vormen in quarantaine worden geplaatst. De quarantaine ontneemt de toegang tot het netwerk van deze eindapparaten. Met als gevolg kunnen eindapparaten wanneer ze besmet zijn met malvare het netwerk niet verder infecteren. 
\newline
\newline
De snelheid waarbij een eindapparaat het netwerk verder infecteert is afhankelijk van de hardware dat wordt gebruikt voor Cisco Identity Sevices Engine en zijn Thrird Party Vendor. In de proef had Cisco Identity Services Engine enkele minuten nodig om het eindapparaat in quarantaine te plaatsen. Tijdens deze enkele minuten kan het eindapparaat het netwerk nog steeds infecteren. Hiervoor kan een toekomstig onderzoek worden uitgevoerd.


!! enquete 


Uit alle resultaten blijkt dat de uitkomst van de uitgevoerde proeven grotendeels in lijn liggen met de resultaten van de enquête. Hierbij biedt dit onderzoek zeker een meerwaarde voor bedrijven die hun eindapparaten beter willen beschermen en beheren tegen interne of externe threads. Het geeft een duidelijk beeld dat integratie van een network access control heel wat voordelen heeft in een bedrijfsnetwerk. Natuurlijk biedt dit onderzoek nog heel wat ruimte voor verder onderzoek. Een denkbare vervolgonderzoek kan een integratie van 'Identity-based network access control' use case in een bedrijfsnetwerk zijn.