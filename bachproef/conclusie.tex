%%=============================================================================
%% Conclusie
%%=============================================================================

\chapter{Conclusie}
\label{ch:conclusie}

In dit onderzoek wordt een antwoord gegeven op de onderzoeksvraag 'Hoe kunnen we een evoluerend bedrijfsnetwerk beter gaan beveiligen door het gebruik van Cisco Identity Services Engine?' Hiervoor is er een studie uitgevoerd waarbij verschillende proeven zijn uitgewerkt die de nood van de Cisco Identity Services Engine use cases naar boven haalt. Vervolgens is de studie aangevuld met een enquête die een opiniepeiling vormt over Cisco Identity Services Engine en zijn use cases.
\newline
\newline
De resultaten van de Port-based en Policy-based network access control use cases bleken positief. Hieruit bleek dat integratie van het desbetreffende use cases het bedrijfsnetwerk op bekabeld niveau robuuster maakt. In het praktisch deel van het onderzoek is er toch een duidelijk verschil op te merken tussen een netwerk met Port-based en Policy-based network access control en een netwerk zonder de use cases. Het is als crimineel veel eenvoudiger om verbinding te maken met het netwerk wanneer de use cases niet werden geïmplementeerd. Hierbij wordt geen bijkomend authenticatie proces gebruikt wanneer men connectie tracht te maken met het netwerk via een Ethernet kabel. Vervolgens betuigt de proef de implementatie van een policy rule waarbij netwerktoegang op bepaalde weekdagen wordt geblokkeerd een gunstig resultaat. Uit deze proef blijkt dat de policy rule een nuttig gegeven is die het netwerk verdedigd op weekend dagen wanneer niemand aanwezig is in het bedrijf.

Uit de resultaten van de Thread-Centric en Policy-based network access control use cases blijkt dat een netwerk wordt beschermd wanneer eindapparaten een bepaald kwetsbaarheid niveau bereiken. Dit resulteert zich in een netwerk waarbij eindapparaten die een kwetsbaarheid vormen in quarantaine worden geplaatst. De quarantaine ontneemt de toegang tot het netwerk van deze eindapparaten. Als gevolg kunnen eindapparaten wanneer ze besmet zijn met malware het netwerk niet verder infecteren. 
\newline
\newline
De snelheid waarbij een eindapparaat het netwerk verder infecteert is afhankelijk van de hardware dat wordt gebruikt voor Cisco Identity Sevices Engine en zijn Thrird Party Vendor. In de proef had Cisco Identity Services Engine enkele minuten nodig om het eindapparaat in quarantaine te plaatsen. In de praktijk kan tijdens deze enkele minuten het eindapparaat het netwerk nog steeds infecteren. Hiervoor kan een toekomstig onderzoek worden uitgevoerd.
\newline
\newline
Uit het onderzoek van de enquête blijkt dat overgrote deel van de specialisten een gelijke waarde en belang hechten aan de integratie van Cisco Identity Services Engine met zijn use cases. Het is echter zo dat Identity-based, Policy-based en Thread-Centric network access control de belangrijkste use cases zijn bij een integratie van Cisco Identity Services Engine. Dit komt voor twee derde overeen met de use cases van dit onderzoek. Vervolgens blijkt uit de resultaten dat 90.9\% van de geënquêteerden vindt dat Cisco Identity Services Engine een logische keuze is ten opzichte van andere network access control producten. Hierbij zou de volle 100\% het product aanbevelen aan anderen. Al bij al zijn de resultaten van de enquête evenredig met de resultaten van de proeven waarbij het belang van Cisco Identity Services Engine en Port-based, Policy-based en Thread-Centric network access control aan bod komt. Wel is duidelijk dat de experten toch een aantal nadelen ondervinden aan het product waarbij toekomstig onderzoek kan hierop inspelen. 
\newline
\newline
Bij het begin van het onderzoek had ik het resultaat van de proeven wel verwacht. Het zijn de resultaten van enquête die toch een onverwachtse wending hebben genomen. Ik had dan ook verwacht dat de Port-based network access control use case een zeer belangrijke use case was, maar de resultaten van de enquête toonden aan dat use case Identity-based network access control de overhand nam. 
\newline
\newline
Als bedrijven opteren om een network access control product te implementeren in het netwerk, dan biedt dit onderzoek zeker een meerwaarde. De integratie zorgt voor een beter beheer en bescherming van eindapparaten in het netwerk tegen interne en externe bedreigingen.  
\newline
\newline
Tot slot biedt dit onderzoek natuurlijk nog heel wat ruimte voor verder onderzoek. Een denkbaar vervolgonderzoek kan een integratie van Identity-based network access control use case in een bedrijfsnetwerk zijn.

  
